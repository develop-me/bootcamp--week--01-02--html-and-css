Anywhere you use a colour

\begin{minted}{css}
article {
    border: 1px solid #fafafa;
    background-color: hsla(146, 56%, 48%, 0.8);
}
\end{minted}

\subsubsection{Colour names}

Red, blue, yellow, aliceblue, firebrick

Transparent

\href{https://en.wikipedia.org/wiki/X11_color_names}{List of colour names}


\subsubsection{Hex codes}

3 part hexadecimal format

Each single or pair represent 0-255 in R, G, or B channel

\texttt{\#FF0000} - R: FF, G: 00, B: 00
\texttt{\#F00} - R: FF, G: 00, B: 00
\texttt{\#EEF4F1} - R: EE, G: F4, B: F1

\subsubsection{rgb()}

Each value 0-255 represents red, green or blue channel

\texttt{rgb(127, 63, 25)}
\texttt{rgb(0, 255, 0)}

\subsubsection{rgba}

Same as rgb but with alpha (opacity/transparency) channel 0-1

\texttt{rgba(255, 255, 255, 1)}
\texttt{rgba(0, 255, 0, 0.5)}

\subsubsection{hsl}

Like rgb, but each value represents *hue*, *saturation* and *lightness* rather than a colour channel

Hue on the colour wheel 0-360, saturation \& lightness percentage

\texttt{hsl(100, 50\%, 50\%)}

\subsubsection{hsla}

Same as before but with alpha channel

\texttt{hsla(200, 20\%, 20\%, 1)}
\texttt{hsla(0, 80\%, 90\%, 0.5)}
