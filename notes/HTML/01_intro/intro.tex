HTML stands for ``HyperText Markup Language''. It's the code the describes the content you see on the World Wide Web.
\\

It's a \textbf{markup language}, based on another language SGML (``Standard Generalized Markup Language'').\footnote{It also has many similarities with XML (``eXtensible Markup Language''), which is itself based on SGML. For a few years the plan was to make HTML fully XML compatible (``XHTML''), but then they realised this would be a bad idea.}

\begin{infobox}{HTML == Content}
    HTML describes the \textit{content} we see on a webpage, CSS is used to \textit{style} it, give it colours and placement. JavaScript is used for \textit{interactivity}, like changing things when a user clicks.
\end{infobox}


\section{Syntax}

``Syntax'' means how the code is written.\footnote{As opposed to ``Semantics'', which is what the code \textit{means}.}

\begin{minted}{html}
    <p class="lede">Here is some text</p>
\end{minted}

Here we write the element \texttt{<p>}, by opening angle brackets, writing the tag name and closing them. We write the content inside and \textit{close} the element by writing the tag name again but adding a forward slash after the first angle bracket.

\begin{infobox}{Closing Elements}
    It's really important to close your HTML elements. Modern browsers will try to do this for you, however if there are multiple unclosed elements it can get confused and cause strange behaviour.
\end{infobox}

There are some elements that close themselves, rather than writing the element again at the end:

\begin{minted}{html}
    <img src="my-image.jpg" alt="A photo of a girl holding a book" />
\end{minted}


\subsection{Attributes}

Inside our elements we have \textbf{attributes}: \texttt{class="lede"}. Attributes give the program more information about the element.
\\

There are lots of these too and they are written by writing the attribute name, followed immediately by an equals sign, then opening up double quotes and writing what we call the \textbf{value} inside.
\\

You can have more than one attribute for an element so we separate them with a space.

\subsection{Nesting elements}

You can put elements inside other elements. This is referred to as \textbf{nesting}.

\begin{minted}{html}
<p class="lede">
    <img src="my-image.jpg" alt="A photo of a girl holding a book" />
</p>
\end{minted}

When we put an element inside another element the one inside is called the child and the one containing the other element is called the parent.
\\

We can put as many elements inside as many other elements as we like!\footnote{Although there are sometimes rules about what should go in what.} This allows us to create a document structure.


\subsection{Whitespace}

This is what we call the spaces in code, like the regular space or tabs at the beginning.
\\

Spaces matter when you write HTML elements because you want to make sure attributes don't merge into each other or the element:

\begin{minted}{html}
<imgsrc="my-image.jpg" />
\end{minted}

Other than the spaces between words and the one after the element's tag name, most spaces in HTML are optional.

\begin{infobox}{Collapsing Whitespace}
    Any whitespace between text content is collapsed into a single space. This allows you to keep your code tidy without worrying about about it looking weird to the user.

    \begin{minted}{html}
        <p>
            Hello.
            How are you?
        </p>
    \end{minted}

    Same as:

    \begin{minted}{html}
        <p>Hello. How are you?</p>
    \end{minted}
\end{infobox}

However, we often add whitespace to make our code easier to read:

\begin{minted}{html}
<p class="lede">
    Here is some text
    <img src="my-image.jpg" alt="A photo of a girl holding a book" />
</p>
\end{minted}

We add some whitespace to the beginning of the line\footnote{Normally four spaces, two spaces, or a tab character – the tab key on your keyboard can be set to do any of these} of the inner element. This lets us see the structure of the document more clearly.
\\

We also add whitespace between attributes to make them easier to discern.


\subsection{Comments}

Comments are parts of code that are ignored by the program interpreting it.
\\

In HTML we write a comment like this:

\begin{minted}{html}
<!-- this is an HTML comment -->
\end{minted}

Comments are useful to describe what the code is doing or to remind yourself of things. They can even be used for documentation.

\section{Basic Structure}

This is the bare minimum you need in your HTML file:

\begin{minted}{html}
<!DOCTYPE html>
<html lang="en">
    <head>
        <title></title>
    </head>
    <body>
        <!--write the content here -->
    </body>
</html>
\end{minted}

There's a ``doctype'' declaration at the top\footnote{For long and complicated historical reasons}, then everything is inside \texttt{html} tags.
\\

The \texttt{<head>} section represents \textbf{metadata} – information about the document. This can be used by the browser for various purposed, but is not part of the visible document.
\\

We add all the content to be shown inside the \texttt{body} element.
\\

There can only be one \texttt{<head>} and \texttt{<body>} element per HTML document


\section{Additional Resources}

\begin{itemize}[leftmargin=*]
    \item \href{https://www.linkedin.com/learning/html-essential-training-4/what-is-html}{What is HTML?}
\end{itemize}
